\documentclass[11pt]{article}
\usepackage{amsmath}
\usepackage{graphicx}
\usepackage{natbib}
\usepackage[dvips, bookmarks, colorlinks=false, breaklinks=true]{hyperref}
\bibpunct{(}{)}{;}{a}{}{,}
\topmargin=-0.5in    % Make letterhead start about 1 inch from top of page 
\textheight=9in  % text height can be bigger for a longer letter
\oddsidemargin=0pt % leftmargin is 1 inch
\textwidth=6.5in   % textwidth of 6.5in leaves 1 inch for right margin
\begin{document}
\title{Simulating Plant Competition in Cycles}
%\author{Yuning Shi (yshi@psu.edu)}
\maketitle
\section{Calculation of leaf area index}
For a given species grwoing alone, the fractional transmitted light by the canopy of species $i$ growing alone is
\begin{equation}\label{eq:LAI}
\tau_i = 1 - F_{0i} \text{,}
\end{equation}
where $F_{0i}$ is the fractional intercepted light by the canopy of species $i$ growing alone.
As
\begin{equation}
\tau_i = \exp\left(-k_i L_i\right) \text{,}
\end{equation}
where $k_i$ is the light extinction coefficient of species $i$, and $L_i$ is the leaf area index of species $i$.
Therefore
\begin{equation}
L_i = \frac{-\log \tau_i}{k_i} \text{.}
\end{equation}

\section{Calculation of radiation competition}
Ignoring the differences in height, the total transmitted light is
\begin{equation}
\tau_T = \exp\left(-\sum_i d_i k_i L_i\right) \text{,}
\end {equation}
where $d_i$ is the planting density (0--1), and $L_i$ is calculated using Equation~(\ref{eq:LAI}).
Then
\begin{equation}
F_T = 1 - \tau_T \text{.}
\end{equation}
The fractional intercepted light by the canopy of species $i$ is
\begin{equation}
F_i = \frac{d_i k_i L_i}{\sum_i d_i k_i L_i} F_T \text{.}
\end{equation}

%\bibliographystyle{ametsoc}

%\bibliography{shi}
\end{document}
